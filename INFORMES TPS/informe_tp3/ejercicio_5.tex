La siguiente secci\'on est\'a dedicada a los \emph{Handlers} o manejadores de las interrupciones. Estos son los c\'odigos que se ejecutan
cuando alguna porci\'on del systema produce un error o llama a una interrupci\'on mejor conocida como syscall o servicios del sistema.\\
Nuestro Kernel cuenta con 3 interrupciones que poseen handlers. Estas fueron mencionadas en el punto 3. Procederemos a explicar cada una de
ellas:
\begin{itemize}
 \item Clock: El clock es una interrupci\'on que se ejecuta cada ciertos ticks de reloj. La misma se encarga de buscar el siguiente selector
de segmento según es especificado en la secci\'on 7. y realizar el salto a dicho selector que puede corresponder a una tarea, una bandera o 
la tarea IDLE.
 \item Teclado: La interrupci\'on de teclado cumple la funci\'on de cambiar el estado del juego.
 \item Servicios o Syscalls: Esta interrupci\'on brinda al systema una serie de servicios o funciones a las tareas:
  \begin{itemize}
   \item Minar: Estas minas destruyen a los tanques que se quieran mover a la posicion donde esta la mina.\\
   Las minas duran una sola vez, es decir, si un tanque es destruido por una mina, ́esta desaparece
(tambien es destruida). Las minas tambien pueden ser destruidas si un misil cae sobre la misma.\\
El unico parmetro de este servicio es la posicion donde colocar la mina. Un tanque puede colocar una mina 
en alguna de las 4 posibles direcciones a su alrededor de la posicion actual donde se encuentra.\\
Una vez llamado este servicio el scheduler desalojara al tanque y ejecutar a la proxima 
tarea mientras se encarga de colocar la mina anti-tanque en su lugar. \\
   \item Lanzar Misil: Los tanques tvenen ca ̃
nones que les permiten disparar balas a sus enemigos, ahora,
como los disparos seran dirigidos a posiciones precisas del mapa, los llamaremos misiles. \\
Dicho esto, este servicio permitir ́a lanzar misiles a posiciones de el mapa direccionadas
de forma relativa a la posicion del tanque. Los valores posibles para x e y no pueden
superar en modulo el tamaño maximo de el mapa, es decir 50.\\
El buffer donde se almacena el misil debe ser un rango dentro del area mapeada por la
tarea, ademas el tamaño del mismo no debe superar los 4096 bytes.\\
El servicio se debe encargar de copiar el buffer del misil dentro al principio de la pagina
fısica idenfiticada por las coordenadas x e y dentro de el mapa. Una vez lanzado el misil,
el scheduler dara paso a la proxima tarea.\\
   \item Mover: El codigo de cada tanque esta mapeado incialmente en dos paginas de memoria a 
   partir de la direccion correspondiente a los 128mb de memoria virtual. A partir de esta
direccion, luego de las dos paginas mapeadas inicialmente, se ir ́an mapeando las 
paginas de memoria f ́ısica que el tanque recorra. Si a la hora de mapear una nueva pagina
fısica, esta esta ya esta mapeada dentro del espacio del tanque, entonces no se mapea
nuevamente.
  \end{itemize}

\end{itemize}

Cabe destacar que las funciones implementadas en C para las syscalls, minar, misil y mover, se encuentran allí dado que manejan p\'aginas
de memoria haciendo que ubicarlas en mmu.c sea lo m\'as conveniente para aprovechar todas las funciones y estructuras utilizadas. 
