Para manejar toda la lógica del juego, los cambios en la pantalla y los mapeos correspondientes tenemos una estructura
llamada Game. Básicamente se encarga de las 3 posibles acciones de una tarea: minar, mover y lanzar misiles.
Tiene guardado internamente una matriz tablero, donde almacenamos el estado del juego actual. 
Ni bien comienza el juego, procedemos a inicializarlo: ponemos todos los valores de la matriz tablero en 0
y distruibuimos las tareas a lo largo y ancho del mapa. Para indicar que el tanque está en cierta posición,
en la matriz tablero ubicamos el número del tanque en la posición correspondiente.

\begin{enumerate}
 \item \textbf{Minar:} Recibimos la posición relativa al tanque que quiere minar y calculamos la posición que le corresponde
 al tablero y con esa, colocamos el valor -1 en el tablero (arbitrario, indica que es una mina) y la dibujamos en pantalla.
 \item \textbf{Misil:} Recibimos la posición relativa a donde lanzar el misil, y además recibimos la dirección desde dónde se copiará el misil (buffer)
 y su tamaño.
 Esta acción es similar a lo que llamamos virus, ya que en caso de caer dicho misil en una parte del código de otro tanque se lo pisará y cambiará,
 generando un posible error en ese tanque cuando vaya a ejecutarse.
 \item \textbf{Mover:} Para realizar esta función del juego utilizamos un struct tanque, el cual tiene su posición columna y fila.
 Lo que hacemos es incrementar o decrementa el valor fila y/o columna según corresponda la dirección pasada como parámetro.
 A su vez dentro de este struct tenemos la ultima dirección virtual mapeada, y se la aumentamos siempre y cuando se mueva a una posicion donde no estaba.
 En caso de haber una superposición con otro tanque no ocurrirá nada (solo se mostrara en pantalla con una x). En cambio si en esa posición 
 el tablero$_{[fila],[columna]}$ es igual a -1, el tanque explotará y dejara de andar, ya que como dijimos anteriormente el -1 indica 
 que hay una mina en esa posición..
\end{enumerate}
