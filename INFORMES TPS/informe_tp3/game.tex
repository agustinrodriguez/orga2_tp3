Para manejar toda la lógica del juego, los cambios en la pantalla y los mapeos correspondientes tenemos una estructura
llamada Game. Básicamente se encarga de las 3 posibles acciones de una tarea: minar, mover y lanzar misiles.
Tiene guardado internamente una matriz tablero, donde almacenamos el estado del juego actual. 
Ni bien comienza el juego, procedemos a inicializarlo: ponemos todos los valores de la matriz tablero en 0
y distruibuimos las tareas a lo largo y ancho del mapa. Para indicar que el tanque está en cierta posición,
en la matriz tablero ubicamos el número del tanque en la posición correspondiente.

\begin{enumerate}
 \item \textbf{Minar:} Recibimos la posición relativa al tanque que quiere minar y calculamos la posición que le corresponde
 al tablero y con esa, colocamos el valor -1 en el tablero (arbitrario, indica que es una mina) y la dibujamos en pantalla.
 \item \textbf{Misil:}dagdsgadsgdsgdsgdsgdsgds
 \item \textbf{Mover:}adgdsgdsgastewrewrwer
\end{enumerate}
