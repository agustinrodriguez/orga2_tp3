Para que el procesador pueda despachar, ejecutar o suspender m\'ultiples tareas, es necesario salvar el estado de las mismas. La 
arquitectura provee mecanismos para esto. El segmento de estado (TSS, Task State Segment), es el que se encarga de almacenar la 
informaci\'on del estado de una tarea.\\

Una tarea est\'a identificada por el selector de segmento de su TSS. Y a su vez la TSS es un segmento, por lo tanto debe estar descripto 
en la GDT junto con los descriptores de segmento de c\'odigo y datos.\footnote{Ver secci\'on 1 para mas informaci\'on sobre esto.}\\

Tenemos 8 tareas y definimos un total de 2 TSS, y otro lo 
dejamos en blanco para la tarea inicial donde se hace el primer salto.\\

Al contar con dos TSS y 8 tareas realizamos cambios de contextos guardando los datos de las tareas con una estructura auxiliar, permiti\'endonos as\'i volver m\'as tarde a esa tarea y no perder la informaci\'on de la 
misma. El cambio de una tss a otra lo realizamos en nuestra estructura de scheduler ejecutando JMP FAR.\\
Por esto mismo es necesario''incializar'' una TSS para que cuando entremos por primera vez la informaci\'on
sea v\'alida. Al momento de inicializar estos segmentos, cada tarea tendran TSS virtualmente id\'enticas, con la excepci\'on del
 eip y pequeños cambios con respecto a las posiciones de las pilas.\\

Como selectores de segmentos de GDT usamos los que definimos para las tareas (es decir, los de nivel 3), y seteamos el RPL en 0x03 para 
evitar un GPE.\\

Una de las grandes ventajas de estar trabajando con direcciones virtuales es que no tenemos que saber la direcci\'on f\'isica exacta de
cada tarea para inicializarlas. Sabemos que todas las tareas comparten ciertas direcciones virtuales, asi que seteamos el directorio de 
p\'agina (CR3) correspondiente a esa tarea y podemos usar direcciones virtuales id\'enticas para todas las tareas. 
\\
