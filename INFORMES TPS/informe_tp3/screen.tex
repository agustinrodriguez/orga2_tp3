La estructura screen tambi\'en es una parte integral del trabajo porque no solo se ocupa de mostrarle
la informaci\'on al usuario, en ella tambi\'en se guardan ciertos datos y se interpreta cierta informaci\'on.
Por esto mismo nos pareci\'o importante agregarle una secci\'on.\\
\\
Nuestra pantalla est\'a separada en 3 gr\'aficos distintos, el mapa, 
la tabla de contexto de tareas (que se encuentra en la parte derecha), la tabla de errores (que se encuentra en la parte inferior derecha) . Todas tienen su propias funciones, y la ventaja que tienen es que pueden ser chequeadas cuantas veces uno quiera presionando la respectiva tecla(Ver interrupcion Teclado).\\
\\
La tabla de contexto de tareas es una estructura que imprime el estado de una tarea, es decir los registros. En caso de que la tarea muera,
esto muestra el ultimo contexto activo antes de morir.\\

