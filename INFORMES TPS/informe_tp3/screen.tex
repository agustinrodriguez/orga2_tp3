La estructura screen tambi\'en es una parte integral del trabajo porque no solo se ocupa de mostrarle
la informaci\'on al usuario, en ella tambi\'en se guardan ciertos datos y se interpreta cierta informaci\'on.
Por esto mismo nos pareci\'o importante agregarle una secci\'on.\\
\\
Nuestra pantalla est\'a separada en 3 gr\'aficos distintos, el mapa, la tabla de errores (que se encuentra a la derecha), la tabla de p\'aginas de tareas (que se encuentra en la parte inferior) . Todas tienen su propias funciones, y la ventaja que tienen es que solo se imprimen mientras una tarea este ocurriendo o cuando hay un error.\\
\\
Tabla errores es una estructura que imprime el estado de una tarea al momento de romperse, es decir los registros. Esto excluye a las instancias en donde una tarea es despejada pero no cay\'o en una interrupci\'on de intel, es decir " cuando pasa malos par\'ametros a la interrupci\'on de servicios", " cuando llama a int 52". \\
\\

