\subsubsection{Global Descriptor Table}
Como ya sabemos, el procesador inicia en ''modo real'', el cual direcciona a 1 MB de memoria y no posee niveles de protecci\'on ni privilegios.\\
Por eso necesitamos que el procesador pase a ''modo protegido'', para direccionar a m\'as memoria y poder manejar distintos niveles de protecci\'on. Nuestro kernel se encargar\'a de hacer esto.\\
\\
Antes de iniciar en modo protegido, es imprescindible tener bien configurado la Tabla de Descriptores Globales, la cual contiene a los descriptores de segmento, con el fin de definir caracter\'isticas de varias \'areas de la memoria.\\
En el enunciado se piden una segmentacion flat, con 4 segmentos que deben direccionar a 1.75 GB: 2 para c\'odigo de nivel 0 y 3 respectivamente, y 2 para datos, de nivel 0 y 3 tambi\'en.\\

La estructura de un descriptor de segmento es la siguiente:\\
\begin{itemize}
  \item L — 64-bit code segment (IA-32e mode only)
  \item AVL — Available for use by system software
  \item BASE — Segment base address
  \item D/B — Default operation size (0 = 16-bit segment; 1 = 32-bit segment)
  \item DPL — Descriptor privilege level
  \item G — Granularity
  \item LIMIT — Segment Limit
  \item P — Segment present
  \item S — Descriptor type (0 = system; 1 = code or data)
  \item TYPE — Segment type
\end{itemize}

Para definir los segmentos que nos requieren, los items importantes son:\\
\begin{itemize}
 \item BASE: este parametro indica el comienzo del segmento. En los 4 casos, este fue 0 ya que se pidió una segmentacion flat.
 \item P: Present, este parametro indica si el segmento est\'a presente en la memoria. El valor en los 4 segmentos es 0x01 ya que efectivamente estaban presentes.
 \item DPL: Nivel de privilegios del descriptor. Dado que se piden dos segmentos de código y dos de datos nivel 0 y nivel3, este 
parametro var\'ia seg\'un cual de estos queremos implementar. Nivel 0 implica DPL = 00b y nivel 3 implica DPL = 11b.
 \item G. Granularity. Este flag indica si el tamaño del descriptor es mayor o menor que 1 Mb. Esto sucede dado que solo se poseen 20 bits para 
indicar el tamaño del segmento. En particular, si G = 1 entonces el valor de los 20 bits ser\'a multiplicado por 4 Kb provocando que con 
 solo 20 bits pueda representar 4Gb de memoria. En nuestro caso queremos un tamaño de 1.75Gb entonces necesitamos G = 1.
 \item Limit: Tamaño del segmento. Va para los 4 segmentos lo mismo.\\
  \indent Tenemos que direccionar a 1.75 GB, que son 1792 MB, que equivalen a 1835008 KB.\\
  \indent Como G vale 0x01, las unidades deben representarse de 4 KB, por eso dividimos por 4.\\
  \indent $\frac{1835008}{4} = 458752$\\
  \indent Pero como la memoria empieza desde el 0, debe ser un n\'umero menos: 458751\\
  \indent $458751 = 0x6FFFF$
 \item Type: Indica si es un segmento de c\'odigo o de datos. Para el segmento de c\'odigo de nivel 0 ponemos el valor de 0x08, indicando 
que es ''Execute only''. para el segmento de c\'odigo de nivel 3 se usa 0x0A, \emph{Read / Execute}. Mientras que para los 2 de datos 
ponemos el valor de 0x02, indicando que son de Read/Write.
\end{itemize}

Tambi\'en se define un segmento que reservado para el \'area de la pantalla en la memoria. Sabemos que empieza en la direcci\'on base
0x000B8000, con un tamaño de 0x0F9F. Dado que se utilizar\'a como un segmemto de datos, su tipo es de Lectura/ Escritura.\\\\
\\
Tambi\'en necesitamos entradas para cada una de las tareas y sus banderas. Es decir, selectores de 
TSS. Estos ser\'an definidos de forma din\'amica y no hardcodeados, bas\'andose en la posici\'on de su respectivo TSS. B\'asicamente cada
una de estas entradas de la GDT para las TSS fue inicada de la siguiente manera:
\begin{itemize}
 \item BASE: Direcci\'on donde fue definido el comienzo de la TSS para cada respectiva tarea.
 \item P: Present. Este flag debe ir seteado para todas las TSS.
 \item DPL: las tareas corren en nivel 3, por lo tanto, el DPL = 3 salvo para las tareas INICIAL e IDLE que deben correr en nivel 0.
 \item limit: Como m\'inimo las TSS tienen un tamaño de 104 bytes es decir 0x67. Esto es como m\'inimo ya que existe la posibilidad de
extender el IO Map Base Address.
 \item type: este es particular. dado que es un tipo de descriptor de segmento, el valor tiene que ser 0x09.
 \item S: este flag determina si el descriptor se refiere a un segmento de c\'odigo o datos o si es de sistema. En este caso como los
descriptores de TSS son de sistema S = 0.
\end{itemize}



\subsubsection{Pasaje a modo protegido}

En funci\'on de pasar a ejecutar en modo protegido el manual de \emph{Intel}\footnote{Ver Intel 64 and IA-32 Architectures Software 
Developer's Manual, Volume 3 System Programming Guide} explicita una serie de pasos que se deben seguir para complir con esto.\\
\begin{itemize}
 \item Habilitar A20. al realizar esto habilitamos el acceso a direcciones superiores a 1 Mb de memoria.
 \item Una vez que tenemos configurada la gdt, guardamos su ubicacion en una variable gdt\_desc. Para que luego la instrucci\'on lgdt 
pueda cargar la direccion de comienzo de la GDT.
 \item Seteamos el flag PE del registro CR0, que indica ''Protected Envirnoment''.
 \item Por \'ultimo para pasar a modo protegido hacemos un jmp al comienzo del segmento de c\'odigo de nivel 0.
 \item Una vez ah\'i acomodamos todos los segmentos apuntando a datos de nivel 0 y seteamos la pila del Kernel en 0x27000 seg\'un lo
indicado por el enunciado.
\end{itemize}
