La siguiente secci\'on est\'a dedicada a los \emph{Handlers} o manejadores de las interrupciones. Estos son los c\'odigos que se ejecutan
cuando alguna porci\'on del systema produce un error o llama a una interrupci\'on mejor conocida como syscall o servicios del sistema.\\
Nuestro Kernel cuenta con 3 interrupciones que poseen handlers. Estas fueron mencionadas en el punto 3. Procederemos a explicar cada una de
ellas:
\begin{itemize}
 \item Clock: El clock es una interrupci\'on que se ejecuta cada ciertos ticks de reloj. La misma se encarga de buscar el siguiente selector
de segmento según es especificado en la secci\'on 7. y realizar el salto a dicho selector que puede corresponder a una tarea, una bandera o 
la tarea IDLE.
 \item Teclado: La interrupci\'on de teclado cumple la funci\'on de cambiar el estado del juego.
 \item Servicios o Syscalls: Esta interrupci\'on brinda al systema una serie de servicios o funciones a las tareas:
  \begin{itemize}
   \item Minar:
   \item Lanzar Misil:
   \item Mover: 
  \end{itemize}

\end{itemize}

Cabe destacar que las funciones implementadas en C para las syscalls, minar, misil y mover, se encuentran allí dado que manejan p\'aginas
de memoria haciendo que ubicarlas en mmu.c sea lo m\'as conveniente para aprovechar todas las funciones y estructuras utilizadas. 
