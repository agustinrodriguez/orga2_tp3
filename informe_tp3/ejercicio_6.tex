Para que el procesador pueda despachar, ejecutar o suspender m\'ultiples tareas, es necesario salvar el estado de las mismas. La 
arquitectura provee mecanismos para esto. El segmento de estado (TSS, Task State Segment), es el que se encarga de almacenar la 
informaci\'on del estado de una tarea.\\

Una tarea est\'a identificada por el selector de segmento de su TSS. Y a su vez la TSS es un segmento, por lo tanto debe estar descripto 
en la GDT junto con los descriptores de segmento de c\'odigo y datos.\footnote{Ver secci\'on 1 para mas informaci\'on sobre esto.}\\

Tenemos 8 tareas y definimos un total de 2 TSS, y otro lo 
dejamos en blanco para la tarea inicial donde se hace el primer salto.\\

Las TSS se actualizan solas con cada JMP Far, permiti\'endonos as\'i volver m\'as tarde a esa tarea y no perder la informaci\'on de la 
misma. Por esto mismo es necesario''incializar'' una TSS para que cuando entremos por primera vez la informaci\'on
sea v\'alida. Al momento de inicializar estos segmentos, cada tarea y su flag tendran TSS virtualmente id\'enticas, con la excepci\'on del
 eip y pequeños cambios con respecto a las posiciones de las pilas.\\

Como selectores de segmentos de GDT usamos los que definimos para las tareas (es decir, los de nivel 3), y seteamos el RPL en 0x03 para 
evitar un GPE.\\

Una de las grandes ventajas de estar trabajando con direcciones virtuales es que no tenemos que saber la direcci\'on f\'isica exacta de
cada tarea para inicializarlas. Sabemos que todas las tareas comparten ciertas direcciones virtuales, asi que seteamos el directorio de 
p\'agina (CR3) correspondiente a esa tarea y podemos usar direcciones virtuales id\'enticas para todas las tareas. 
\\
Las TSS de las flags son un caso particular por dos razones. La primera es que el eip no es un valor que sepamos de antemano, sino que 
depende de cada tarea. Al final de cada tarea hay un offset guardado, que sum\'andolo a 0x4000 0000 nos da la direcci\'on virtual 
de la funcion flag. La segunda es que queremos que flag se comporte como una funcion tradicional, es decir, que corra siempre del 
principio hasta el final (o ser interrumpida). En pocas palabras, no nos interesa la posici\'on donde estuvo la \'ultima corrida, sino
que nos interesar\'ia volver siempre al comienzo.\\
Para resolver esto generamos dos funciones que definen el eip del TSS de un flag forma din\'amica, y que deben ser corridas antes de 
saltar a un flag. Por un lado tenemos la funci\'on fetch\_eip, que se encarga de averiguar el offset de la bandera buscando la tarea 
en la memoria (recordar que las tareas "navegan" y en teor\'ia podrian mutar), y la funci\'on reset\_eip, que escribe este dato dentro 
de la TSS.\\
